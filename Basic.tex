\documentclass{beamer}
% \usetheme{Frankfurt}
% \usecolortheme{default}
\usepackage{hyperref}
\usepackage[utf8]{inputenc} % this is needed for german umlauts
\usepackage[english]{babel}
\usepackage[T1]{fontenc}    % this is needed for correct output of umlauts in pdf

\title{Practical of CV:HCI}
\subtitle{WS 2016 / 2017 \# 4 (2.477745)}
\author[Martin\,Thoma \& Bettina\,Weller\& Yang\,Zhang]
{%
  \texorpdfstring{
    \begin{columns}%[onlytextwidth]
      \column{.30\linewidth}
      \centering
      Martin Thoma\      \href{mailto:info@martin-thoma.de}{info@martin-thoma.de}
      \column{.30\linewidth}
      \centering
      Bettina Weller\\    \href{bettinaweller@web.de}{bettinaweller@web.de}
      \column{.30\linewidth}
      \centering
      Yang Zhang\\    \href{yang.zhang@student.kit.edu}{yang.zhang@student.kit.edu}
    \end{columns}
  }
  {John Doe \& Jane Doe}
}
\date{16. January 2017}
\subject{Computer Science}

\begin{document}
\setbeamertemplate{navigation symbols}{}
\frame{\titlepage}

\begin{frame}{1st Assignment: Color-based skin classifier}
    \begin{itemize}
        \item Idea: Histogramm-based approach
        \item HSV color space
        \item $256 \times 256$ bins (H and S, V is ommitted)
        \item $p_\text{Skin}(H, V) = \frac{\text{\# skin pixels in bin } (H, V)}{\text{\# skin pixels in bin } (H, V) + \text{\# non-skin pixels in bin } (H, V)}$
        \item Don't forget to multiply $p_\text{Skin}(H, V)$ with 256.
        \item[$\Rightarrow$] $F_1 = 0.875242$
    \end{itemize}
\end{frame}

\begin{frame}{2nd Assignment: Person detector}
    \begin{itemize}
        \item Idea: Fiddle around with parameters
        \item \texttt{hog.blockStride = Size(16, 28)}
        \item \texttt{hog.nbins = 20}
        \item \texttt{hog.winSigma = 30}
        \item polynomial svm kernel with margin
        \item[$\Rightarrow$] $F_1 = 0.976542$ 
    \end{itemize}
\end{frame}

\begin{frame}{3rd Assignment: Train a FACE Similitude Measure}
    \begin{itemize}
        \item Idea: PCA + Distance measure
        \item Distance measure
        \begin{itemize}
            \item Euclidean distance was good
            \item Cosine distance was worse than Euclidean distance
        \end{itemize}
        \item Preprocessing: Grayscale + Crop image to middle 60\%; 
        \item 150 components, only first 200 images because of server speed
        \item[$\Rightarrow$] $1- \text{EER} = 0.641847$ 
    \end{itemize}
\end{frame}

\end{document}
